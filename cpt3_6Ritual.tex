\section{Das Ritual}
\subsection{Ritualsystem: Beschwörung \& Bannung Azaluths}
Das Ritualsystem basiert auf Fragmenten, die die Spieler im Laufe des Abenteuers entdecken (z. B. Amulett, Bannformel, Siegelkachel, Kinderlied, Kultmantra etc.). Es gibt zwei mögliche Rituale: Beschwörung (versehentlich oder absichtlich) und Bannung (schwierig, fragmentarisch).
\subsubsection{Ritual der Beschwörung (das der Kult verfolgt)}
\paragraph{Voraussetzungen:}
\begin{itemize}
\item Kultkreis (z. B. Siebenmünde oder Krypta)
\item Blutopfer oder „freiwillige Öffnung“ (z. B. ein Kultist, der sich aufschlitzt)
\item Aktiviertes Kultamulett
\item Sprechformel (fragmentarisch: z. B. „sieh mich, hör mich, lass mich werden“)
\item Niemand darf den Namen aussprechen – es muss durch Musik, Symbole oder Gedanken gerufen werden
\end{itemize}
\paragraph{Ablauf:}
\begin{enumerate}
\item Opfer bringt sich innerhalb des Kreises um → das Blut muss geometrisch fließen (z. B. in Spiralen)
\item Die sieben äußeren Steine „summen“ (spielerisch: hörbar oder metaphorisch)
\item Träger des Amuletts spricht das Mantra rückwärts (Wurf auf Okkultismus gegen 75)
\item Azaluth manifestiert sich als „unsichtbarer Druck“ → Lichter verlöschen, Stimmen verstummen
\end{enumerate}
\paragraph{Effekte:}
\begin{itemize}
\item Wahnsinnsproben für alle im Umkreis von 10m
\item Spieler*innen hören ihre eigenen Gedanken aus fremden Mündern
\item Wetter schlägt um, Tiere verschwinden
\end{itemize}
\subsubsection{Ritual der Bannung (fragmentarisch, schwer)}
\paragraph{Benötigt:}
\begin{itemize}
\item Die Bannformel (Handout 6) → vollständige Version nötig (Wissen + Okkultismus > 70)
\item Die Siegelkachel (Handout 8) – darf nicht zerbrochen sein
\item Ein „Reinherziger“ (z. B. Kind, Unschuldiger) muss außerhalb des Rituals stehen und den Namen des Wesens nicht kennen
\item Das Ritual muss schweigend durchgeführt werden
\end{itemize}
\paragraph{Ablauf}
\begin{enumerate}
\item Teilnehmer bilden einen Kreis um das Zentrum → jeder muss eine Komponente halten (z. B. Amulett, Kachel, Pergament)
\item Eine Person spricht die Formel leise – bei Lautstärke > Flüstern: Ritual bricht ab
\item Willenswurf jedes Teilnehmers gegen 60 → bei Scheitern: der Charakter „wird gehört“ (SL-Effekt)
\item Bei Erfolg: ein leuchtendes Zeichen erscheint → Wesen wird für 7 Tage gebannt
\item Bei kritischem Erfolg (W100 $\leq$ 5): Zeichen wird permanent → Kult verliert Macht
\end{enumerate}
\paragraph{Fehlversuche}
\begin{itemize}
\item Amulett zerbricht → Azaluth „blickt“ in die Welt
\item Jemand spricht den Namen laut aus → Ritual schlägt um → Beschwörung
\end{itemize}

\subsection{Finale Szene: Der Kreis von Siebenmünde}
\paragraph{Aufbau:}
\begin{itemize}
\item Nebel zieht auf, die Kultisten (z. B. Frater Dorian) und vielleicht ein „leise Besessener“ treffen dort ein
\item Spieler können:
    \begin{enumerate}
    \item Die Beschwörung verhindern (z. B. mit Ritual der Bannung)
    \item Das Ritual sabotieren (z. B. Blut durch Wein ersetzen, Steine versetzen)
    \item Den Kult gewähren lassen – Azaluth „erblickt die Welt“ → Horror-Ende
    \item Mit dem Kult verhandeln → verrückter Pakt (eine Erinnerung opfern)
    \end{enumerate}
\end{itemize}

\subsubsection{Lösungswege}
\paragraph{1. Der klassische Kampf}
\begin{itemize}
\item Spieler greifen die Kultisten an
\item Kämpfe im Kreis (z. B. gegen Tessa, Marius)
\item Gefahr: Während des Kampfes könnte das Ritual durch Zufall vollzogen werden (Blut auf Steinen etc.)
\end{itemize}
\paragraph{2. Das Gegenritual}
\begin{itemize}
\item Mit Siegelkachel, Bannformel, Amulett
\item Optional: Lise (Kind) als „Reinherzige“ in Sicherheit bringen
\item Spannung: Einer der Spieler muss sich opfern oder Wahnsinn riskieren
\end{itemize}
\paragraph{3. Psychologisches Finale}
\begin{itemize}
\item Spieler erkennen, dass der Kult nur eine Idee ist
\item Azaluth kann nicht zerstört, sondern nur „vergessen“ werden
\item Nur durch kollektives Verschweigen wird er gebannt → jeder Charakter muss eine Lüge erzählen und sie glauben
\end{itemize}
\paragraph{4. Der Horror-Ausgang}
\begin{itemize}
\item Kult siegt
\item Spieler erleben Vision der neuen Welt
\item Bonus: Die Charaktere „existieren weiter“, aber ohne Identität – Azaluth nimmt ihnen ihre Namen
\end{itemize}
\paragraph{Zeitdruck vs. Vorbereitung auf das Finale Problem:} Das Ritual kann durch viele Dinge ausgelöst werden – SCs könnten überrumpelt werden, bevor sie alle nötigen Bann-Komponenten gesammelt haben. Lösung: Gib der SL eine „weiche Dramaturgie“ an die Hand. Erst wenn mindestens zwei Schlüsselobjekte gefunden wurden oder SCs die Ritualstätte betreten, beginnt die Eskalation. Alternativ kann Dorian das Ritual verzögern – er will prüfen, ob die SCs würdig/schwach sind.
