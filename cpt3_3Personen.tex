\section{Personen}
\subsection{Elias Grimm – Totengräber und ehemaliges Kultmitglied}
\paragraph{Beschreibung:}
Hagerer Mann mit fahlem Blick und eigenartigen Umgangsformen. Lebt am Rande des Friedhofs. Hat früher dem Kult angehört, sich aber losgesagt. Kennt viele der alten Zeichen und Rituale. Redet häufig in halben Sätzen.
\paragraph{Werte:}
\begin{itemize}
\item Stärke: 35
\item Geschicklichkeit: 48
\item Intelligenz: 77
\item Charisma: 42
\item Mut: 60
\item Heimlichkeit: 55
\item Überreden: 30
\item Einschüchtern: 61
\item Wissen (okkult): 82
\item Wahrnehmung: 69
\end{itemize}
\paragraph{Ausrüstung:}
\begin{itemize}
\item Kleines schwarzes Gebetsbuch mit eigenhändig veränderten Passagen
\item Dolch mit eingravierter Spirale (Relikt des Kults, benutzt er nicht mehr)
\item Eingetrocknete Kräuterbündel
\end{itemize}
\paragraph{Dialogbeispiele:}
\begin{itemize}
\item „Tote reden nicht, wenn du fragst. Aber wenn du zuhörst...“
\item „Die Zeichen bleiben. Im Nebel. Im Blut. In uns.“
\item „Ich habe ihn gesehen. Nicht mit Augen.“
\end{itemize}
\paragraph{Geheimnis:}
Er kennt den alten Bannspruch gegen Azaluth (Fragment). Zögert aber, ihn weiterzugeben – aus Angst, dass das Wesen dann „erwacht“.
\newpage
\subsection{Gerlinde Rasp – Wirtin des „Fetten Rehs“}
\paragraph{Beschreibung:}
Ehemals Klosterschwester, jetzt überdrehte Wirtin. Herzhaft, schnippisch, mit einer Schwäche für Rum und skurrile Aberglauben. Tut unbeteiligt, hört aber alles.
\paragraph{Werte:}
\begin{itemize}
\item Stärke: 30
\item Geschicklichkeit: 39
\item Intelligenz: 64
\item Charisma: 76
\item Heimlichkeit: 50
\item Menschenkenntnis: 68
\item Kochen: 89
\item Wissen (Kirche): 58
\item Lügen: 52
\end{itemize}
\paragraph{Ausrüstung:}
\begin{itemize}
\item Schlüsselbund (inkl. Dachboden)
\item Uralte Bibel mit durchgestrichenen Psalmen
\item Rezept für „Geisterwein“ (beruhigendes Mittel aus Kräutern und Alkohol)
\end{itemize}
\paragraph{Dialogbeispiele:}
\begin{itemize}
\item „Was habt ihr denn mit der Stirn? So’n Sumpfdruck gehabt?“
\item „Wenn ihr was hört, ignoriert es. Wenn es euch anfasst – nicht mein Problem.“
\item „Im Dachboden? Da wohnt’s. Ich hab Frieden gemacht mit ihm.“
\end{itemize}
Geheimnis:
Sie hat den Kult nie vergessen. Auf dem Dachboden hält sie ein „altes Relikt“ verborgen, aus Angst, es könnte benutzt werden.
\newpage
\subsection{Bürgermeister Jost Vollmer – nervöser Lokalpolitiker}
\paragraph{Beschreibung:}
Ordnungsliebend, bürokratisch und panisch bemüht, dass niemand „etwas Dummes“ im Moor anstellt. Hat selbst Angst, aber will die Stadt nicht in Panik versetzen.
\paragraph{Werte:}
\begin{itemize}
\item Stärke: 28
\item Intelligenz: 71
\item Charisma: 63
\item Einschüchtern: 35
\item Lügen: 75
\item Schreiben / Lesen: 94
\item Strategie: 68
\item Menschenkenntnis: 45
\item Wissen (lokal): 59
\end{itemize}
\paragraph{Ausrüstung:}
\begin{itemize}
\item Amtskette, Schlüssel zum Archiv
\item Siegelstempel, Notizbuch mit seltsamen „Wachträumen“
\item Karte des Moors mit eingezeichnetem „verbotenen“ Bereich
\end{itemize}
\paragraph{Dialogbeispiele:}
\begin{itemize}
\item „Ich rate Ihnen dringend, sich nicht in Dinge einzumischen, die... lange ruhen sollten.“
\item „Das Moor ist Sperrgebiet. Seuchenrisiko. Offiziell.“
\item „Ihr meint... der Stein? Unsinn. Alter Aberglaube. Natürlich.“
\end{itemize}
\paragraph{Geheimnis:}
Vollmer hatte selbst Visionen vom Wesen – spricht aber nicht darüber. Will niemandem schaden, aber ist bereit, zu lügen und sogar Spieler*innen aufzuhalten.
\newpage
\subsection{Lise – Kind mit der „Fantasie“}
\paragraph{Beschreibung:}
9 Jahre alt. Lebt bei ihrer Großmutter, streunt viel umher. Beobachtet Dinge, die Erwachsene nicht sehen wollen. Gilt als „sonderbar“. Manche sagen, sie sei vom Geist ihrer Mutter berührt worden (diese starb im Moor).
\paragraph{Werte:}
\begin{itemize}
\item Stärke: 18
\item Geschicklichkeit: 62
\item Intelligenz: 65
\item Wahrnehmung: 91
\item Heimlichkeit: 72
\item Wissen (Moorpfade): 67
\item Mut: 20
\end{itemize}
\paragraph{Ausrüstung:}
\begin{itemize}
\item Stofftier mit Knopfaugen (kann plötzlich Dinge „sehen“)
\item Kreidezeichnungen von Wesen mit „vielen Mündern“
\item Ein Lied, das sie summt – ein altes Ritual in Kinderreimform
\end{itemize}
\paragraph{Dialogbeispiele:}
\begin{itemize}
\item „Ich hab sie gehört. Sie sagt, du sollst nicht ins Wasser gehen.“
\item „Der Stein hat gestern geschlafen. Heute nicht mehr.“
\item „Opa sagt, ich soll den Namen nicht sagen. Dann kommt er raus.“
\end{itemize}
\paragraph{Geheimnis:}
Sie ist unbewusst mit der Essenz Azaluths verbunden. Kein „Besessenheit“, sondern ein Echo. Wenn sie getötet oder geopfert würde, könnte das Wesen erwachen.
\newpage
\subsection{Meister Brandt – alter Holzschnitzer}
\paragraph{Beschreibung:}
Griesgrämiger, wuchtiger Mann mit einer ruhigen, düsteren Aura. Redet nicht gern. Seine Werkstatt enthält seltsame Skulpturen mit zu vielen Gliedmaßen.
\paragraph{Werte:}
\begin{itemize}
\item Stärke: 61
\item Geschicklichkeit: 80
\item Intelligenz: 54
\item Kunsthandwerk (Schnitzen): 94
\item Wahrnehmung: 52
\item Heimlichkeit: 33
\item Wissen (Runen): 44
\end{itemize}
\paragraph{Ausrüstung:}
\begin{itemize}
\item Ein geschnitztes „Gesicht ohne Gesicht“
\item Schnitzmesser mit Bronzegriff (nicht von ihm gefertigt)
\item Runentafel mit sieben Symbolen
\end{itemize}
\paragraph{Dialogbeispiele:}
\begin{itemize}
\item „Hände können lügen. Holz nicht.“
\item „Ich schnitze nur, was ich träume.“
\item „Du willst wissen, was fehlt? Schau nicht hin. Spür’s.“
\end{itemize}
\paragraph{Geheimnis:}
Einst von Azaluth „berührt“. Seine Kunst überträgt das, was er nicht begreift – sie ist prophetisch.
\newpage
\begin{enumerate}
\item Gegner und mögliche (zweifelhafte) Verbündete im Kult,
\item NSCs, die auf falsche Fährten führen (bewusst oder unbewusst),
\item „Leise Besessene“, deren Werte und Verhalten sich subtil verändern – ideal für Paranoia, Misstrauen und Wendungen.
\end{enumerate}
   

\subsection*{1. Gegner / Verbündete im Kult}
\paragraph{Hinweis:} Der Kult ist eher ein „Netzwerk aus Gedankenechos“ als eine klare Organisation. Die Mitglieder sind keine Robenträger, sondern Träger eines alten „Wissens“, das flüstert. Einige glauben an Erlösung, andere an Macht.

\subsection{Frater Dorian (Kultführer im Verborgenen)}
\paragraph{Rolle:} Finaler Antagonist oder tragischer Verbündeter
\paragraph{Deckidentität:} Einsiedler in einer verlassenen Kapelle am Moor
\paragraph{Motiv:} Glaubt, Azaluth könne die Welt „erlösen“, indem es alle Trennung (Körper / Geist / Zeit) aufhebt
\paragraph{Auftreten:} Sanft, höflich, spricht wie ein Priester
\paragraph{Werte:}
\begin{itemize}
\item Intelligenz: 88
\item Charisma: 81
\item Okkultismus: 94
\item Täuschung: 78
\item Mut: 45
\item Widerstand gegen geistige Beeinflussung: Null (da aufgegeben)
\end{itemize}
\paragraph{Fähigkeit:} Kann Spieler geistig manipulieren (z. B. „Gedankensaat“ – bei Berührung muss ein Wurf auf Willenskraft > 60 erfolgen, sonst wird man empfänglich für Suggestion)
\paragraph{Zitat:}
„Er ist kein Dämon. Er ist das, was war, bevor wir Schatten wurden.“
\newpage
\subsection{Tessa Gramlich – Die Sünderin}
\paragraph{Rolle:} Kultanhängerin, aber potentielle Informantin
\paragraph{Deckidentität:} Näherin, lebt allein in einem abgelegenen Stall
\paragraph{Merkmale:} Redet wirr, aber enthält wichtige Fragmente (z. B. Ortsnamen, Rituale)
\paragraph{Werte:}
\begin{itemize}
\item Heimlichkeit: 80
\item Wahrnehmung: 77
\item Okkultismus: 65
\item Täuschung: 70
\item Stärke: 30
\end{itemize}
\paragraph{Besonderheit:} Hat Visionen. Wenn sie Spieler*innen berührt, spüren diese für 10 Sekunden, wie es ist, im Moor zu „ertrinken“. Danach erhalten sie eine Vision oder ein Symbol.
\paragraph{Zitat:}
„Sie haben mir die Haut genommen. Nur außenrum. Das Innere flüstert noch.“
\paragraph{Geheimnis:} Will sterben, damit Azaluth sie „ganz“ macht – könnte sich opfern oder ein Spielerkind verführen.

\newpage
\subsection*{2. NSCs auf falscher Fährte}

\subsection{Vater Rudel – Der Prediger}
\paragraph{Rolle:} Fanatisch religiöser Wanderprediger, der den Kult anklagt – aber zu viel projiziert
\paragraph{Verhalten:} Schreit auf dem Marktplatz, beschuldigt Unschuldige
\paragraph{Werte:}
\begin{itemize}
\item Charisma: 72
\item Überreden: 65
\item Einschüchtern: 60
\item Wissen (Bibel): 81
\item Okkultismus: 28 (veraltet, falsch)
\end{itemize}
\paragraph{Zitat:}
„Das Biest trägt viele Masken – eure Kinder, eure Weiber, selbst euer Spiegel!“
Falsche Fährte: Spielleitung kann ihn Spieler*innen beschuldigen lassen – etwa Lise, Gerlinde oder Brandt. Lenkt vom wahren Ursprung ab.

\newpage
\subsection{Ulrich Mörschel – Der Holzfäller}
\paragraph{Rolle:} Außenseiter mit seltsamen Angewohnheiten – aber kein Kultist
\paragraph{Verhalten:} Wohnt im Wald, redet mit Tieren, trägt Knochenkette
\paragraph{Werte:}
\begin{itemize}
\item Stärke: 85
\item Naturkunde: 68
\item Heimlichkeit: 33
\item Kochen (Räucherfleisch): 91
\item Charisma: 14
\end{itemize}
\paragraph{Zitat:}
„Ich ess kein Fleisch, das mich anschaut.“
Falsche Fährte: Man findet bei ihm okkulte Symbole – die er aber gegen Waldgeister nutzt. Keine Verbindung zum Kult. Wenn angegriffen, zieht er sich zurück oder kämpft.

\newpage
\subsection*{3. „Leise Besessene“}
Diese Personen wirken harmlos – aber sie sind in Gedanken „offen“ für Azaluth. Sie verändern sich im Lauf des Spiels subtil. Die Spieler bemerken es vielleicht – oder eben nicht.

\subsection{Alte Mutter Wede}
\paragraph{Rolle:} Großmutter von Lise. Trinkt Kräutertee, spricht zu Vögeln.
\paragraph{Veränderung:} Beginnt, im Schlaf zu sprechen – sagt Sätze, die niemand kennen kann.
\paragraph{Werte zu Spielbeginn:}
\begin{itemize}
\item Intelligenz: 58
\item Charisma: 60
\item Wissen (Heilpflanzen): 72
\item Wahrnehmung: 52
\end{itemize}
\paragraph{Veränderte Werte (nach Besessenheit):}
\begin{itemize}
\item Okkultismus: +30
\item Wahrnehmung: +20 (sie sieht Dinge, die nicht da sind)
\item Charisma: -20 (wirkt geistesabwesend)
\end{itemize}
\paragraph{Spielmechanik:}
Wenn ein Spieler mit ihr spricht, kann sie auf einmal Details sagen, die nur der Spielercharakter wissen kann – dies erzeugt Gänsehaut und Paranoia.

\newpage
\subsection{"Fratzel", Elias’ Katze}
\paragraph{Rolle:} Normal wirkende Katze – bis sie den Spielern folgt.
\paragraph{Veränderung:} Ihre Augen leuchten bei Nacht. Sie schläft nie.
\paragraph{Effekt:}
\begin{itemize}
\item Wenn sie im Raum ist, misslingt jede zweite Wahrnehmungsprobe automatisch.
\item Optional: Sie überträgt Träume vom Kult an Spieler*innen
\end{itemize}
\paragraph{Verhalten:} Sitzt immer auf dem höchsten Möbelstück und starrt in die Ecke.

\newpage
\subsection{Knecht Marius (Gasthaus)}
\paragraph{Rolle:} Jung, höflich, hilfsbereit.
\paragraph{Veränderung:} Wird wortkarg, blinzelt nie, wirkt „glatter“.
\paragraph{Anzeichen:} Beginnt Sätze mit „Sie sagen...“ oder „Wir hören euch auch“
\paragraph{Effekt:}
\begin{itemize}
\item Seine Werte bleiben gleich, aber sein Verhalten ändert sich schleichend
\item Bei direkter Konfrontation flieht er – Richtung Moor
\end{itemize}

\newpage
\subsection{Dialogbaukasten}
Es folgt ein Dialogbaukasten für die wichtigsten NSCs im Uhlenburg-Abenteuer. Die Dialogoptionen sind modular gestaltet – mit Fragen, Reaktionen und Ergebnissen, je nach Vorgehen der Spieler. Die Dialoge sind nur beispiele und keine festen Vorgaben. Sie sollen nur eine Idee für neue SL sein, wie Interaktionen mit NSCs gestaltet werden können:
\begin{itemize}
\item Gesprächsstil \& Auftreten
\item Was sie sagen, wenn man sie direkt fragt
\item Was sie nur andeuten, wenn man sie überzeugt
\item Was sie lügen oder verschweigen
\item Was sie zufällig preisgeben (z. B. bei Smalltalk oder Beobachtung)
\end{itemize}
\paragraph{Gerlinde Vogt} (Witwe des ehemaligen Bürgermeisters)
\begin{itemize}
\item Stil: Vorsichtig, leicht abwesend, schwerhörig – aber klug.
\item Ort: Eigenes Haus, voller alter Gegenstände und Kräuterkram.
\item Wissen: Früheres Siegel, Bannformel (Fragment), Vergangenheit des Kultes
\end{itemize}
→ Direkt gefragt nach dem Kult / Gerüchten\\
„Ach Kind, das war alles vor meiner Zeit... aber meine Großmutter sprach manchmal im Schlaf. Von Dingen ohne Form. Und Wasser, das sich erinnert.“\\
→ Spieler erhält vage Hinweise auf wiederkehrende „Wellen“, evtl. generationales Trauma.\\
→ Bei guter Überreden-/Vertrauensprobe
„Ich hab was gefunden. In der Dachbalkenritze. Etwas mit lateinischen Zeichen. Hab’s weggeschlossen, damit’s mir die Nacht nicht verdirbt…“\\
→ Gibt Pergamentfragment preis (Teil der Bannformel).\\
→ Was sie verschweigt:\\
Sie weiß, dass ihr Mann mit dem damaligen Einsiedler über ein „Siegel“ stritt. Sie fühlt sich mitschuldig.\\
→ Lässt sich nur unter Tränen in der Nacht entlocken.\\
→ Zufällige Bemerkung:\\
„Ich hasse den Wind von Osten. Da riecht’s immer nach altem Eisen.“
\paragraph{Knecht Marius} (leise Besessener, Gasthaus)
\begin{itemize}
\item Stil: Ruhig, höflich, zunehmend unheimlich.
\item Ort: Gasthof „Zur Linde“ (Schankraum, Hof)
\item Wissen: Siebenmünde, der Kultkreis, Azaluths „Flüstern“
\end{itemize}
→ Smalltalk / „Wie geht’s dir?“
„Ich träum nicht mehr, das ist gut. Die Stille ist besser als die Fragen.“\\
→ Bei Fragen nach dem Moor oder seltsamen Dingen:\\
„Siebenmünde ist kein Ort, es ist ein Kreis. Oder eine Richtung. Oder beides.“\\
→ Wenn man ihn beschuldigt / konfrontiert:\\
„Warum denn? Ich hab euch doch nicht geweckt…“ (Lächelt – viel zu lang.)\\
→ Lüge / Tarnung:\\
„Ich weiß nichts. Ich geh kaum raus.“\\
→ Beobachtung (Wahrnehmung > 60): Seine Stiefel sind voller schwarzem Moorschlick.
\paragraph{Magda} „die Mager“
\begin{itemize}
\item Stil: Kräuterweise, direkt, aber mit Aberglaube getränkt.
\item Ort: In der Hütte am Waldrand. Immer Kräuter am Trocknen.
\item Wissen: Kinderlieder = Bannritual, Bedeutung von Pflanzen, Tierverhalten
\end{itemize}
→ Bei Interesse an Kräutern / Heilung:\\
„Für Albträume hilft Frauenmantel mit Linde. Aber wenn’s aus’m Wasser kommt, hilft nur Stille.“\\
→ Spieler können hier Hinweise auf das „Schweigen“ im Ritualkontext erahnen.\\
→ Nach dem Bann gefragt (z. B. „Kann man sowas aufhalten?“)\\
„Schweigen ist stärker als Rufen. Ich kenn ein Lied. Ist alt. Hab’s nie zu Ende gelernt...“\\
→ Singt halbe Strophe → Teil des Banntexts (Verstärkung möglich mit Lise).\\
→ Was sie nicht sagt:\\
Sie war einst Teil eines Gebetskreises, der scheiterte. Verlor dabei ihr Kind.\\
→ Nur bei empathischem Zugang (z. B. SC erzählt von Verlust o.ä.)
\paragraph{Lise} (Kind)
\begin{itemize}
\item Stil: Unruhig, verträumt. Zeichnet viel. Spricht in Bildern.
\item Ort: Zuhause oder auf dem Feld hinter dem Dorf.
\item Wissen: Visionen, Spirale, das Gesicht im Wasser
\end{itemize}
→ Allgemein:\\
„Ich hab sie gesehen. Sie singen durch meine Finger.“\\
→ Meint: Ihre Zeichnungen entstehen „von selbst“.\\
→ Wenn man ihr eine Zeichnung zeigt:\\
„Die hab ich auch geträumt. Aber da war noch ein Arm mehr.“\\
→ Hinweis auf Wesen mit wandelbarer Gestalt\\
→ Wenn sie sich wohlfühlt:\\
„Ich mag das Wasser, aber es mag mich nicht. Es will, dass ich still bin.“\\
→ Spieler können ahnen: Das Ritual verlangt Schweigen (→ Bannmechanik)
\paragraph{Vater Rudel} (falsche Fährte)
\begin{itemize}
\item Stil: Fanatisch, laut, beschuldigend.
\item Ort: Auf dem Marktplatz oder vor der Kirche
\item Wissen: Falsch, aber ein paar Dinge stimmen zufällig
\end{itemize}
→ Predigt-Schnipsel:\\
„Das Tier ist unter uns! Mit fremder Stimme redet es durch unsere Kinder!“\\
→ Spieler denken an Lise oder Marius.\\
→ Wenn Spieler zuhören oder provozieren:\\
„Siebenmünde ist ein Tor zur Hölle! Ihr müsst die Steine salzen!“\\
→ Irreführung: Steine salzen hat keinen Effekt.\\
→ Wahrer Kern (zufällig):\\
„Der Name soll nicht gesprochen werden. Er gehört dem Atem.“\\
→ Passt zur Bannbedingung („nicht laut aussprechen“)
\paragraph{Alte Mutter Wede} (später besessen)
\begin{itemize}
\item Stil: Warm, vergesslich, dann plötzlich unheimlich präzise
\item Ort: Haus neben Lise
\item Wissen: Fast alles – sobald die Besessenheit einsetzt
\end{itemize}
→ Vorher (normal):\\
„Ich bin alt. Ich vergesse. Aber der Nebel... der ist älter.“\\
→ Besessenheitsphase (PL ansagen durch Rollenverhalten):\\
„Ich weiß, was du verlierst, wenn du ihn ansiehst.“\\
→ Erzählt den SCs plötzlich Geheimnisse aus ihrer Vergangenheit.\\
→ Gefährlicher Satz:\\
„Sag mir deinen Namen. Dann geb ich dir den meinen.“\\
→ Wer das tut, riskiert mentale Verschmelzung mit Azaluth.
\paragraph{Frater Dorian} (Kultführer)
\begin{itemize}
\item Stil: Höflich, messianisch, nicht aggressiv – ein guter Gesprächspartner
\item Ort: Kapelle im Moor oder Ritualplatz
\item Wissen: Fast alles, aber mit eigener ideologischer Brille
\end{itemize}
→ Wenn SCs ihn zur Rede stellen:\\
„Ihr glaubt, ihr haltet das Dunkel draußen? Es war schon immer in euch.“\\
→ Doppeldeutig, psychologisch → schürt Selbstzweifel\\
→ Wenn SCs diplomatisch sind:\\
„Ihr könnt gehen. Oder zuhören. Oder euch erinnern. Es ist egal – Azaluth kennt euch schon.“\\
→ Drohung ohne Lautstärke\\
→ Offenbart nur auf direkte Bitte / Opfergabe:\\
„Ihr wollt ihn bannen? Ihr seid schon Teil des Liedes.“