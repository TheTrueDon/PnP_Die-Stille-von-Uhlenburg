\chapter{Modul - Mentale Stabilität und Wahnsinn}
\section{Mechanik}
\subsection{Einführung eines Willenskraft-Attributs} 
Jeder SC bekommt beim Erstellen zusätzlich zu seinen Talenten den Wert Willenskraft (z. B. zwischen 30 und 70). Dieser Wert wird wie ein Talent behandelt.\\
Typisch: 
\begin{itemize}
\item Bauernkind: 30
\item Soldat: 50
\item Priester, Mystiker: 60–70
\item Wahnsinnsanfällige SCs: 20–40
\end{itemize}
\subsection{Mentale Stabilitätspunkte (MSP)}
Optional kannst du auch einen Zustandswert einführen:
\begin{itemize}
\item Jeder SC hat Mentale Stabilität = 10 Punkte (Standardwert)
\item Bei Schock, Begegnungen mit Azaluth, Albträumen etc. verliert man 1–3 MSP.
\item Bei 0 MSP: der SC bricht mental zusammen / erleidet eine psychische Störung
\end{itemize}
Heilung von MSP:
\begin{itemize}
\item Durch Ruhe (1 Punkt pro Nacht ohne Störung)
\item Gespräche mit vertrauten NSCs
\item Magische oder rituelle Reinigung
\item Erfolg bei wichtigen moralischen Proben
\end{itemize}
\subsection{Wahnsinnsproben (Willenskraft-Wurf)}
Ablauf:
\begin{itemize}
\item Du beschreibst als SL eine psychisch belastende Szene (z. B. Azaluth erscheint, Traumvision, Tod eines NSCs etc.).
\item Der SC muss einen Wurf auf Willenskraft (W100 unterhalb des Werts) machen.
\item Bei Misserfolg: Wahnsinnseffekt tritt auf + evtl. 1W3 MSP-Verlust
\end{itemize}
\subsection{Effekte bei Wahnsinn / Zusammenbruch}
\begin{center}
\begin{tabular}{ll}
\toprule
\textbf{MSP Zustand} & \textbf{Effekt} \\ 
\midrule
10 - 6 & Keine / leichte Unruhe (Nachteile bei Konzentration) \\ 
\midrule
5 - 3 MSP & Albträume, Verwirrung, -10 auf soziale Talente \\ 
\midrule
2 - 1 MSP & Zwangshandlungen, Paranoia, Sprachverlust, Halluzinationen \\ 
\midrule
0 MSP & Totaler Zusammenbruch: SC wird handlungsunfähig, bewusstlos oder besessen (SL-Entscheidung) \\ 
\bottomrule
\end{tabular} 
\end{center}
\subsection{Beispiele für Wahnsinnseffekte (bei Misserfolg)}
\begin{center}
\begin{tabular}{ll}
\toprule 
\textbf{Art des Effekts} & \textbf{Beschreibung} \\ 
\midrule 
Panik & SC flieht aus Szene, verliert 1 Runde \\ 
\midrule 
Stille / Sprachverlust & SC kann nicht sprechen oder nur flüstern \\ 
\midrule 
Realitätsverzerrung & SC hält Begleiter für Feinde, Wahrnehmungswahn \\ 
\midrule 
Lähmung & SC friert ein, kann 1–2 Runden nicht handeln \\ 
\midrule 
Besessenheit & SL übernimmt kurzzeitig SC-Handlungen \\ 
\bottomrule 
\end{tabular} 
\end{center}
\subsection{Talente, die bei Stabilität helfen}
\begin{center}
\begin{tabular}{ll}
\toprule
\textbf{Talent} & \textbf{Nutzen im System} \\ 
\midrule 
Willenskraft & Widersteht Wahnsinnseffekten \\ 
\midrule 
Wissen (Okkultes) & Besseres Verständnis $\rightarrow$ geringerer Schockwert \\ 
\midrule 
Empathie / Menschenkenntnis & Kann anderen bei Wahnsinn helfen \\ 
\midrule 
Meditation / Gebet / Glaube & Regeneration von MSP durch Rituale \\ 
\bottomrule
\end{tabular} 
\end{center}
\textbf{Beispiel:}\\
Szene: Azaluth manifestiert sich. Die SCs spüren, wie das Moor selbst zu atmen scheint.\\
SL: Macht bitte alle einen Wurf auf Willenskraft. Der Schwierigkeitsgrad ist hoch (Erfolg unter 50).\\
$\bullet$ SC 1 würfelt 62: Fehlschlag $\rightarrow$ verliert 2 MSP, bekommt den Wahnsinnseffekt \textit{Sprachlosigkeit} für 10 Minuten.\\
$\bullet$ SC 2 würfelt 33: Erfolg $\rightarrow$ bleibt ruhig.
\subsection{Wahnsinns-/Stabilitäts-Tabelle}
Zufallstabelle für Fehlschläge bei Willenskraftproben oder bei Stabilitätsverlust:\\
Wenn ein SC eine Wahnsinnsprobe nicht besteht oder MSP auf 5 oder weniger sinken, würfle auf dieser Tabelle (W20) oder wähle passend zur Szene.\\
\textbf{W20 Wahnsinnseffekte:}
\begin{small}
\begin{center}
\begin{tabular}{lll}
\toprule
\textbf{Wurf} & \textbf{Effektname} & \textbf{Beschreibung / Auswirkung} \\ 
\midrule 
1 & Fluchtreflex & Der SC flieht instinktiv aus der Szene, rennt in zufällige Richtung (1W6 Runden) \\ 
\midrule 
2 & Einfrieren & Der SC ist für 1W4 Runden handlungsunfähig, starrt nur ins Leere \\ 
\midrule 
3 & Sprachlosigkeit & Der SC kann nicht mehr sprechen (bis Beruhigung / Szenenende) \\ 
\midrule 
4 & Weinkrampf & Der SC beginnt unkontrolliert zu weinen; -20 auf soziale Talente für 1 Stunde \\ 
\midrule 
5 & Schlafsucht & Der SC will sich hinlegen – sofortiger Erschöpfungswurf (Erschöpfung +1) \\ 
\midrule 
6 & Zwangshandlung & Wiederholt eine nutzlose Handlung (z. B. zählt Steine, murmelnd beten) – 1W6 Minuten \\ 
\midrule 
7 & Todesangst & Bei Bedrohung immer Rückzug oder Defensive; +20 auf Flucht, -20 auf Angriff \\ 
\midrule 
8 & Halluzination & Der SC sieht etwas Schreckliches (z. B. Azaluth in Gefährten); SL beschreibt \\ 
\midrule 
9 & Wortsalat & Der SC redet wirres Zeug, kann nicht sinnvoll kommunizieren (10 Minuten) \\ 
\midrule 
10 & Gedächtnislücke & Der SC vergisst 1W4 Stunden (oder wichtige Info) \\ 
\midrule 
11 & Kältegefühl & SC friert trotz Wärme – Zähneklappern, Zittern, -10 auf alle körperlichen Talente \\ 
\midrule 
12 & Lichtempfindlich & -20 auf Wahrnehmung bei Tageslicht für 1W6 Stunden \\ 
\midrule 
13 & Misstrauen & Der SC traut 1 anderen SC nicht mehr und sagt das deutlich \\ 
\midrule 
14 & Ohrensausen & SC hört Azaluths Flüstern, kann keine leisen Geräusche mehr deuten \\ 
\midrule 
15 & Besessenheit (leicht) & SL beschreibt, wie der SC kurzzeitig wie fremdgesteuert agiert \\ 
\midrule 
16 & Lachanfall & Unkontrollierbares Lachen für 1W6 Minuten \\ 
\midrule 
17 & Fokussierung & SC fixiert sich auf einen NSC oder Objekt, will es beschützen oder zerstören \\ 
\midrule 
18 & Schwächeanfall & -2 auf Stärke-basierte Würfe, zitternde Hände \\ 
\midrule 
19 & Stimmenhören & SC hört Stimmen (Azaluth, Kind, NSC) – muss dem nachgehen oder antworten \\ 
\midrule 
20 & Wiederkehr der Stille & Azaluth nimmt dem SC für 24 Stunden seine Stimme, Gedanken oder Sinn \\ 
\bottomrule
\end{tabular} 
\end{center}
\end{small}

\newpage
\section{Spielerhilfe}
\paragraph{Mentale Stabilität – Übersicht}
\begin{center}
\begin{tabular}{ll}
\toprule
\textbf{Regelbestandteil} & \textbf{Erklärung} \\ 
\midrule 
Willenskraft-Wert & Talentwert (z. B. 30–70), W100-Probe gegen Wahnsinn oder mentale Effekte \\ 
\midrule 
Mentale Stabilität (MSP) & Startwert: 10. Bei Albträumen, Kultkontakt etc. sinkend. Bei 0 = Zusammenbruch \\ 
\midrule 
Willenskraft-Probe & SL kündigt \textit{Wahnsinnsprobe} an. W100 $\leq$ Willenskraft = Erfolg \\ 
\midrule 
MSP-Verlust & 1W3 bei Fehlschlag, selten mehr \\ 
\midrule 
Erholung von MSP & +1/Nacht guter Schlaf, +2 durch Rituale, +1 durch Zuspruch (soziale Probe) \\ 
\midrule 
Boni durch Bannrituale & Temporär +20 auf Willenskraftproben gegen Azaluth-Einfluss \\ 
\bottomrule
\end{tabular} 
\end{center}

\paragraph{Effekte bei sinkender MSP}
\begin{center}
\begin{tabular}{ll}
\toprule
\textbf{MSP-Wert} & \textbf{Effekt (Richtlinie)}\\
\midrule 
10–6 & leichte Verstörung (Albträume, Unruhe)\\
\midrule 
5–3 & -10 auf soziale Proben, Albträume, erste Halluzinationen\\
\midrule 
2–1 & Paranoia, Zwang, Wahnvorstellungen\\
\midrule 
0 & Totalausfall: Panik, Ohnmacht oder dauerhafter Schaden (SL-Entscheid)\\
\bottomrule
\end{tabular} 
\end{center}

\paragraph{Beispielhafte Proben im Spiel}
\begin{center}
\begin{tabular}{lll}
\toprule
\textbf{Szene} & \textbf{Probe auf Willenskraft?} & \textbf{MSP-Verlust?}\\
\midrule 
Azaluth erscheint im Nebel & Ja & 1W3\\
\midrule 
SC schläft im Bannkreis ohne Schutz & Ja (Traum) & 1W2\\
\midrule 
SC sieht besessenes Kind & Optional & 1\\
\midrule 
SC wird im Ritual angesprochen & Ja & 1W4 (wenn scheitert)\\
\midrule 
Gespräch mit Kultanhänger & Nur bei starker Belastung & 0–1\\
\bottomrule
\end{tabular} 
\end{center}

\newpage
\section{SL-Leitaden}
\paragraph{Grundlagen für Willenskraft \& Wahnsinn}
\begin{itemize}
\item Willenskraft ist ein Talent (W100 $\leq$ Wert $\rightarrow$ Erfolg).
\item Mentale Stabilitätspunkte (MSP): Alle SC starten mit 10 MSP.
\item Bei Wahnsinnsquellen (z. B. Azaluth, Albträume, Beschwörung) $\rightarrow$ Willenskraftprobe.
\item Bei Fehlschlag: 1W3 MSP-Verlust + ein Effekt von der Wahnsinnstabelle.
\end{itemize}

\paragraph{Mentale Stabilitäts-Stufen}
\begin{center}
\begin{tabular}{ll}
\toprule
\textbf{MSP} & \textbf{Effekt (Richtlinie)}\\
\midrule 
10–6 & leichte Irritation (Unwohlsein, Albträume, innere Unruhe)\\
\midrule 
5–3 & -10 auf soziale / geistige Proben, Angst, Misstrauen\\
\midrule 
2–1 & Wahnvorstellungen, Stimmenhören, emotionale Kontrollverluste\\
\midrule 
0 & Zusammenbruch: Ohnmacht, Besessenheit, völlige Handlungsunfähigkeit\\
\bottomrule
\end{tabular} 
\end{center}
\paragraph{Wahnsinnsprobe – Ablauf (Standardregel)}
\begin{enumerate}
\item SL beschreibt belastende Szene $\rightarrow$ „Wahnsinnsprobe“ aufrufen
\item SC würfelt W100 $leq$ Willenskraft $\rightarrow$ Erfolg
\begin{itemize}
\item Bei Erfolg: SC bleibt stabil
\item Bei Fehlschlag: 1W3 MSP-Verlust + 1 Effekt von der Wahnsinnstabelle (W20)
\end{itemize}
\end{enumerate}

\paragraph{W20 Wahnsinns-Effekte (Auswahl bei Fehlschlag)}
\begin{center}
\begin{tabular}{lll}
\toprule
\textbf{Wurf} & \textbf{Effektname} & \textbf{Beschreibung}\\
\midrule 
1 & Fluchtreflex & SC flieht instinktiv (1W6 Runden)\\
\midrule 
2 & Einfrieren & Handlungsunfähig 1W4 Runden\\
\midrule 
3 & Sprachlosigkeit & Kann nicht sprechen bis Szenenende\\
\midrule 
4 & Weinkrampf & -20 auf soziale Talente für 1 Stunde\\
\midrule 
5 & Schlafsucht & Sofortige Erschöpfung +1\\
\midrule 
6 & Zwangshandlung & Zählt, summt, betet unkontrolliert (1W6 Min.)\\
\midrule 
7 & Halluzination & Sieht etwas Unwirkliches (Azaluth, Schatten)\\
\midrule 
8 & Stimmenhören & Flüstern im Kopf $rightarrow$ Handlungserzwingung möglich\\
\midrule 
9 & Besessenheit (leicht) & SL übernimmt für 1–2 Runden Kontrolle\\
\midrule 
10 & Lachanfall & Kann nicht ernst agieren (1W6 Min.)\\
\bottomrule
\end{tabular} 
\end{center}

\paragraph{Beispiele für Wahnsinnsquellen}
\begin{center}
\begin{tabular}{lll}
\toprule
\textbf{Situation} & \textbf{Willenskraft-Probe?} & \textbf{MSP-Verlust bei Fehlschlag}\\
\midrule 
Direkter Blickkontakt mit Azaluth & Ja & 1W3\\
\midrule 
Träume mit Bannlied / Schreien aus dem Moor & Ja & 1W2\\
\midrule 
Besessenes Kind spricht plötzlich & Ja (mittelschwer) & 1\\
\midrule 
Teilnahme an unfertigem Ritual & Ja (schwer) & 1W4\\
\midrule 
Anblick eines Kultsymbols (vollständig) & Optional & 1\\
\bottomrule
\end{tabular} 
\end{center}

\paragraph{Stabilisierung / Erholung von MSP}
\begin{center}
\begin{tabular}{lll}
\toprule
\textbf{Methode} & \textbf{Erholung MSP}\\
\midrule 
Gute Nachtruhe & +1\\
\midrule 
Rituelle Reinigung & +2\\
\midrule 
Vertrauen + Zuspruch & +1 (soziale Probe)\\
\midrule 
Erfolgreiche Bannhandlung & +1 bis +3\\
\midrule 
Einnahme bestimmter Kräuter & +1 (zeitweise), ggf. -10 auf andere Proben\\
\bottomrule
\end{tabular} 
\end{center}

\paragraph{Talente, die helfen können}
\begin{itemize}
\item Willenskraft: direkte Proben gegen Wahnsinn
\item Okkultes Wissen: Kann helfen, Quellen zu erkennen (Wahnsinn vermeiden)
\item Empathie / Heilkunde Seele: Beruhigen anderer SCs
\item Ritualkunde / Glauben: Vorbereitung schützender Rituale (+20 Bonus für andere SCs)
\end{itemize}

\paragraph{SL-Tipps für Anwendung}

\begin{itemize}
\item Nicht inflationär einsetzen – Azaluths Präsenz soll selten, aber intensiv sein.
\item Wahnsinnsproben gezielt nutzen: bei Albträumen, Ritualen, Visionen, Beschwörungen.
\item Spielerwahl zulassen – Spieler können Wahnsinn freiwillig zulassen für Vorteile (z.B. Vision erhalten, Hinweis bekommen).
\item Folgen auch erzählerisch einbinden – Halluzinationen können Hinweise oder Irreführung sein.
\end{itemize}