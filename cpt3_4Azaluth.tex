\section{Azaluth – Das Schweigende Wesen des Moors}
\subsection{Wesen und Erscheinung}
Azaluth ist ein uraltes, urzeitliches Wesen, das tief im Moor verankert ist — eine lebendige Verkörperung von Verfall, Stille und vergessener Erinnerung. Es ist weder ganz physisch noch vollständig geistig; vielmehr ein Zwischenreichwesen, dessen Form sich ständig in nebligen Schwaden auflöst und neu zusammensetzt.
\subsubsection*{Körper}
Azaluth hat eine humanoide Grundform, doch scheint sein Körper aus tiefschwarzem, feuchtem Moor und organischem Schlamm zu bestehen. An manchen Stellen blitzen uralte, moosbedeckte Knochenfragmente und wurzelförmige Gebilde hervor, die sich langsam pulsierend bewegen. Sein Körper wirkt schwer, zugleich jedoch fast schwebend und diffus, als wäre es halb in der Welt, halb im Sumpf gefangen.
\subsubsection*{Gesicht}
Das Gesicht Azaluths ist von einem Schleier aus dichtem Nebel verhüllt, der beständig in sanften Wellen wabert. Nur schemenhafte Andeutungen von Augen sind erkennbar — tiefglühende, kalte Lichtpunkte, die den Beobachter zugleich hypnotisieren und erschauern lassen. Der Mund ist nie wirklich sichtbar, doch aus der nebligen Tiefe dringt ein Flüstern, das keiner Sprache entspricht, eher eine uralte Resonanz im Geist.
\subsubsection*{Gliedmaßen}
Seine Arme enden in langen, wurzelartigen Fingern, die im Dunkel fast wie Tentakel wirken. Diese greifen langsam, formlos und können sowohl trösten als auch festhalten — je nachdem, ob Azaluth gewollt oder gehasst wird.
\subsection{Wesen und Natur}
\subsubsection*{Lebensraum}
Azaluth ist untrennbar mit dem Moor verbunden, das es bewohnt. Es ist eine Manifestation von Stille, Vergessen und dem unheimlichen Schweigen des Wassers und der Erde. Wo es sich aufhält, zieht sich das Leben zurück, Tiere verstummen, und die Zeit scheint langsamer zu fließen.
\subsubsection*{Einfluss auf Menschen}
Das Wesen nährt sich von der Angst und der Sehnsucht nach Vergessen. Es spricht nicht durch Worte, sondern durch Träume, Halluzinationen und schattenhafte Erscheinungen. Menschen, die ihm zu nahe kommen, erleben einen tiefen inneren Schwebezustand zwischen Realität und Wahnsinn.
\subsubsection*{Kommunikation}
Azaluth sendet keine klaren Botschaften, sondern flüstert mit dem Wind und dem Moorwasser. Es ist, als ob die Seele des Moors selbst zu sprechen beginnt – diffus, rätselhaft und nie direkt.
\subsubsection*{Symbolik}
Azaluth steht für das Schweigen vor der Wahrheit, das vergessene Wissen und die Angst vor dem Unausweichlichen. Es fordert Respekt vor dem Unbekannten und Opfern – oft in Form von Ritualen, die das Schweigen bewahren oder brechen können.
\subsection{Rolle im Kult}
Der Kult verehrt Azaluth als „Den Flüsternden“, das uralte Wesen, das über das Moor wacht.\\
Die Rituale des Kultes dienen dazu, Azaluth zu besänftigen, aber auch, um sein Schweigen in die Welt zu tragen – ein Schweigen, das angeblich vor einer drohenden Katastrophe schützen soll.\\
Azaluth selbst wirkt dabei nicht als aggressiver Dämon, sondern als natürliche, aber unheimliche Macht, die weder gut noch böse ist, sondern einfach „ist“.
\subsection{Stimmung und Atmosphäre um Azaluth}
Nebel und Kälte folgen seinem Erscheinen.
Eine bedrückende Stille breitet sich aus, nur das ferne Tropfen von Wasser oder leises Rascheln von Schilf ist zu hören.
Zeitweise wirkt Azaluth wie eine Manifestation kollektiver Ängste und Geheimnisse – die düstere Seele einer Region, die niemand anrühren will.
\subsection{Begegnungen mit Azaluth}
\paragraph{Erste Wahrnehmung (vage und unheimlich)}
Die Luft wird plötzlich kälter, dichter Nebel zieht auf, obwohl kein Wind zu spüren ist.
Ein dumpfes, feuchtes Geräusch klingt in der Ferne – als ob schwere Schritte im Morast versinken.
Ein schwaches, kaum hörbares Flüstern dringt an dein Ohr, aber du kannst die Worte nicht verstehen.
Du hast das Gefühl, beobachtet zu werden, doch du siehst nichts.
\paragraph{Sichtbare Manifestation (selten, flüchtig)}
Vor dir formt sich ein dunkler Schatten, eine humanoide Gestalt, deren Körper wie aus Moor und Wurzeln geformt scheint.
Nebelschwaden umhüllen das Wesen, sein Gesicht bleibt verschwommen, nur zwei kalte Lichtpunkte blicken dich an.
Die Gestalt bewegt sich langsam, fast schwebend, und streckt einen wurzelartigen Arm aus, als wollte sie dich berühren, doch der Kontakt entgleitet dir wie Wasser.
Ein Flüstern klingt in deinem Geist, eine uralte Resonanz, die dich zugleich beruhigt und ängstigt.
\paragraph{Begegnung im Moor (direkter Kontakt)}
Der Boden unter deinen Füßen wird matschig und kühlt bis ins Mark. Plötzlich verstummt jegliches Tierleben – kein Vogel zwitschert, kein Frosch quakt.
Die Schwaden um dich verdichten sich, und aus dem Nebel tritt Azaluth hervor. Sein Blick trifft deinen, und du fühlst eine Mischung aus tiefer Traurigkeit und unendlicher Kälte.
Seine Stimme ist kein Laut, sondern ein Dröhnen in deinem Inneren: eine Botschaft von Vergessen und Schweigen.
Deine Sinne beginnen zu verschwimmen, Zeit und Raum lösen sich auf, und für einen Moment bist du eins mit dem Moor – doch der Preis ist dein Bewusstsein.
\subsection{Täume und Visionen mit Azaluth}
\paragraph{Traum des Flüsterns}
Du findest dich an einem Seeufer wieder, das Wasser ist still wie Glas. Unter der Oberfläche siehst du dunkle, wurzelartige Formen, die sich bewegen.
Plötzlich steigen dünne Nebelschwaden empor, formen ein verschwommenes Gesicht mit kalten, leuchtenden Augen.
Du hörst ein Flüstern, nicht in Worten, sondern in Gefühlen – Angst, Vergessen, Sehnsucht.
Du versuchst zu sprechen, doch kein Laut verlässt deinen Mund.
\paragraph{Vision des Schweigens}
In deinem Traum bist du gefangen in einem endlosen Moor, das sich wie eine lebende Masse um dich schließt.
Die Geräusche verstummen, deine Gedanken werden dumpf, als würde das Moor selbst deinen Geist einfangen.
Eine Gestalt aus Nebel und Moor tritt hervor, und obwohl sie dich nicht berührt, fühlst du eine uralte Macht, die dein Schweigen fordert.
Der Traum endet mit dem Echo eines singenden Windes, der eine Melodie trägt, die du zu kennen glaubst, aber nicht singen darfst.
\paragraph{Warnung im Schlaf}
Du erwachst schweißgebadet, und die letzten Worte eines uralten Liedes hallen in deinem Kopf nach.
Es ist eine Melodie, die von Schweigen, Opfern und Vergessen erzählt.
Im Zimmer ist alles still, doch ein Gefühl von Beobachtung lässt dich nicht los.
Vielleicht ist es das Flüstern von Azaluth, das dich warnt – oder dich ruft.
\newpage
\subsection{Spielmechaniken für Begegnungen \& Täume}
\subsubsection*{Wahnsinnsproben (Willenskraft-Wurf)}
\paragraph{Situation:}
Wenn die SCs Azaluth direkt begegnen oder eine besonders intensive Vision/Traum erleben.
\paragraph{Mechanik:}
\begin{itemize}
\item Probenwurf auf Willenskraft (z.B. 40+) mit W100
\item Bei Misserfolg erleidet der SC temporäre mentale Effekte (siehe unten).
\item Bei Erfolg widersteht er dem Einfluss, aber die Erfahrung bleibt bedrückend.
\end{itemize}

\subsubsection*{Angst \& Beeinträchtigung (für Begegnungen)}
\paragraph{Auslöser:}
Sichtbare Manifestation oder direkte Konfrontation im Moor.
\paragraph{Effekt bei misslungener Probe:}
\begin{itemize}
\item SC erleidet Erschöpfung +1 wegen Panik.
\item Der SC hat für die nächste Szene -10 auf Wahrnehmungs- und Kampfproben (Wahrnehmung, Reaktion, Kämpfen).
\item Optional: SC kann fliehen, bekommt aber Nachteile für Gruppenkoordination.
\end{itemize}

\subsubsection*{Wahrnehmungsverlust / Desorientierung (für Träume)}
\paragraph{Situation:} Nach einem intensiven Traum mit Azaluth.
\paragraph{Effekt:}
\begin{itemize}
\item Für eine Stunde ist der SC geistig abwesend, kann keine komplexen Proben durchführen (z. B. Magie, Überreden).
\item SC bekommt -20 auf Intelligenz- und Willenskraft-Proben wegen geistiger Ermattung.
\item SC fühlt sich innerlich leer und schweigsam.
\end{itemize}

\subsubsection*{Langfristige Effekte: Nachwirkungen des Kontakts}
\begin{itemize}
\item \textbf{Schlafstörungen:} SC kann für mehrere Nächte nicht richtig schlafen, was zu -1 auf Willenskraft und Reaktion tagsüber führt.
\item \textbf{Albträume: }SC hat bei Schlaf Würfe auf Wahnsinn (z. B. alle 2 Nächte), bei Misserfolg folgt Verwirrung oder Angstzustände (SL entscheidet).
\item \textbf{Schweigen: }Manche SCs sprechen plötzlich weniger oder gar nicht, besonders wenn sie das Schweigen von Azaluth nicht verstanden haben.
\end{itemize}

\subsubsection*{Rituale zum Schutz vor Azaluth (optional)}
\begin{itemize}
\item Ein erfolgreiches Bannritual (z. B. mit Bannformel + Bannlied) kann SCs für kurze Zeit vor mentalen Effekten schützen (z. B. +20 auf Willenskraft-Proben gegen Wahnsinn).
\item Das Tragen des Kult-Amuletts kann kurzfristig vor direktem Einfluss schützen, bringt aber Nachteile im sozialen Umgang (Misstrauen).
\end{itemize}
\newpage