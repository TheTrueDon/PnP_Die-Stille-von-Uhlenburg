\section{Orte}
\subsection{Zentrale Orte}
    \paragraph{Stadt Uhlenburg}
	\begin{itemize}
    \item Marktplatz mit Pranger und Brunnen, Gasthaus „Zum Fetten Reh“, Schmiede, Bürgermeisterhaus
     \item Atmosphäre: Verschlossen, mißtrauisch. Manche ahnen mehr.
     \item  NSCs:
		\begin{itemize}
         \item Wirtin Gerlinde (ehemals Klosterschwester, trinkt zu viel)
         \item Bürgermeister Vollmer (kontrollierend, versteckt Angst)
         \item Totengräber Elias (weiß mehr, redet in Rätseln)
         \item Dorfkind Lise (phantasiert, aber spricht die Wahrheit)
        \end{itemize}
    \end{itemize}
    \paragraph{Altes Benediktinerkloster (Ruine)}
	\begin{itemize}
     \item Verschüttete Gruft unter der Kapelle
     \item Gebetsbuch mit eigenartigem Zusatzkapitel (in Blut geschrieben)
     \item Feuchtigkeit, kalter Wind, leise „Stimmen“ beim Betreten
    \end{itemize}
    \paragraph{Das Uhlenmoor}
	\begin{itemize}
     \item Lebendig wirkend: Nebel bewegt sich gegen den Wind, Vögel stürzen tot ab
     \item „Der schreiende Stein“ (versunkene Kultstätte)
     \item Altes Boot mit eingekerbten Symbolen
     \item Sinkende Fundamente eines Schuppens mit Ritualrelikten
    \end{itemize}

\subsection{Uhlenburg}
\paragraph{Größe:} ca. 300–500 Einwohner
\paragraph{Lage:} Am Rande eines alten Moores, in einem bewaldeten Talkessel gelegen. Ein kleiner Fluss („Uhl“) fließt durch den Ort.
\paragraph{Gliederung:}
\begin{itemize}
\item Zentrale Marktstraße (Nord-Süd-Achse), von der kleinere Gassen abzweigen
\item Stadtmauer (halbverfallen im Westen, besser erhalten im Osten)
\item Klosterruine im Norden auf leichtem Hügel
\item Moor im Westen, über schmalen Steg erreichbar
\item Südtor führt zu einer alten Poststraße
\item Friedhof im Südosten bei der Kapelle
\end{itemize}

\paragraph{Übersicht: Hauptbereiche}
\begin{enumerate}
\item Der Marktplatz (Zentrum)
\item Gasthaus „Zum fetten Reh“ (Ostseite des Platzes)
\item Haus des Bürgermeisters Vollmer (am Südrand des Platzes)
\item Schmiede und Werkstätten (Westlich des Platzes)
\item Klosterruine (nördlich des Stadtkerns, erhöht)
\item Friedhof und Totengräberhütte (Südosten)
\item Der Uhlensteig (Pfad ins Moor, beginnt westlich beim alten Stadttor)
\item Wohntviertel / Gassen („Gickelgasse“, „Talgang“, „Klostertreppe“)
\item Verlassene Häuser im Westen („Hinterzeile“)
\item Alter Stadtturm / Archiv (nördlicher Mauerrest)
\end{enumerate}

\paragraph{1. Der Marktplatz}
\begin{itemize}
\item Aussehen: Breiter, unebener Platz mit Brunnen in der Mitte. Pflaster teilweise gebrochen, der Pranger daneben ist morsch.
\item Nutzung: Marktstände an Markttagen (Mo \& Do), Versammlungen, öffentliche Ansprachen
\item Stimmung: Schlecht beleuchtet, Tauben, Geruch nach Räucherfisch und Asche
\item Gerüchte: Alte Leute behaupten, bei Nacht singe der Brunnen
\end{itemize}
\paragraph{2. Gasthaus „Zum fetten Reh“}
\begin{itemize}
\item Wirtin: Gerlinde, ehem. Klosterschwester, heute kettenrauchend und schnapsliebend
\item Stil: Holzbau, windschief, mit präpariertem Rehkopf über der Tür
\item Innen: Grober Tresen, fünf Tische, ein Kamin, ein Billard-ähnliches Spiel („Hufwerfen“)
\item Zimmer: Drei Gästezimmer, Dachboden wird nicht vermietet
\item Besonderheiten: Auf einem Balken eingeritzt: „Schweigt, wenn’s zischt“
\item Gerücht: Auf dem Dachboden wohne ein „Klopfer“, der nachts schreit
\end{itemize}

\paragraph{3. Bürgermeisterhaus}
\begin{itemize}
\item Bewohner: Bürgermeister Vollmer + Haushälterin Lotte
\item Bauweise: Stein, zweigeschossig, neoklassizistisches Eingangsportal
\item Innen: Büro mit Archiv, versperrte Truhe, Landkarte mit Moor
\item Besonderheit: Hängt ein Porträt von „Graf Uldemar“ (187 Jahre alt), dessen Augen „folgen“
\item Gerücht: In Vollmers Keller sei ein Altar – oder ein Weinkeller, je nach Erzähler
\end{itemize}

\paragraph{4. Schmiede \& Werkstätten}
\begin{itemize}
\item Schmiedin: Marga, kräftig, misstrauisch, hilft nur gegen Bezahlung
\item Werkstätten: Holzschnitzer (Meister Brandt), Gerber, Weberin
\item Stimmung: Rauch, rußig, oft Lärm
\item Fundort: Alte Glocke mit unleserlicher Inschrift liegt im Hof
\item Gerücht: Der Gerber war früher Totengräber – und redet im Schlaf mit „ihm“
\end{itemize}

\paragraph{5. Klosterruine (Nordhang)}
\begin{itemize}
\item Zugang: Über „Klostertreppe“ – 89 moosige Stufen
\item Struktur: Kapelle, zerstörter Glockenturm, verschlossene Krypta
\item Sicht: Von hier aus sieht man die Moorfläche
\item Gerücht: In der Ruine singen Stimmen – sogar bei Windstille
\item Besonderheiten: Altar zeigt ein Lamm mit sechs Augen
\end{itemize}

\paragraph{6. Friedhof \& Totengräberhütte}
\begin{itemize}
\item Gräber: Teils verwittert, manche umgestoßen
\item Hütte: Schiefe Baracke, innen gepflegt, viele Bücher, Kräutersträuße
\item Elias (Totengräber): Lebt hier, spricht mit Gräbern
\item Geheimnis: Grabstein mit fremden Schriftzeichen (ein Kultsymbol)
\item Gerücht: Elias’ Katze ist 30 Jahre alt – oder gar keine Katze
\end{itemize}

\paragraph{7. Der Uhlensteig (ins Moor)}
\begin{itemize}
\item Pfad: Beginnt beim westlichen Stadttor, teils Bohlenweg, teils Matsch
\item Gefahren: Erdrutsche, Nebel, Orientierungslosigkeit
\item Hinweis: Am Rand liegt ein „verlassenes Bootshaus“
\item Gerücht: Wer dem Nebel folgt, sieht sein eigenes Gesicht im Wasser, aber tot
\end{itemize}

\paragraph{8. Gassen \& Viertel}
\begin{itemize}
\item Gickelgasse: Hühner, dicht gedrängt, Schreie von Kindern
\item Talgang: Hangabwärts, dort wohnen „Randexistenzen“, viel Gesocks
\item Klostertreppe: Führt zur Ruine
\item Besonderheit: Viele Türen mit seltsamen Zeichen eingeritzt (gegen das „Feuchte“)
\end{itemize}

\paragraph{9. Die Hinterzeile (Verlassene Westseite)}
\begin{itemize}
\item Verlassen: Nach dem großen Brand vor 20 Jahren, nie wiederaufgebaut
\item Gebäude: Eingestürzt, durchwühlt, manchmal Unterschlupf für Streuner
\item Fundmöglichkeit: Verbranntes Tagebuch eines Mönchs mit Symbol
\item Gerücht: Im Keller des alten Lagerhauses lebt „der Sumpfmann“
\end{itemize}

\paragraph{10. Alter Stadtturm \& Archiv}
\begin{itemize}
\item Zustand: Einst Wachturm, heute eingestürzt, nur Unterbau begehbar
\item Innen: Altes Archiv, Regale mit Pergamentrollen, Siegelwachs
\item Gefahr: Einsturzgefahr bei starker Erschütterung
\item Besonderheit: Geheimgang zum Kloster (verschüttet)
\item Gerücht: Uralte Karte mit „gebanntem Kreis“ liegt dort versteckt
\end{itemize}
\newpage
\subsection{Die Kultstätten im Uhlenmoor}
Das Moor westlich von Uhlenburg war einst heiliges Land – lange vor der Christianisierung. Der Kult um „Azaluth“ – eine fremdartige, nicht-personifizierte Macht, die als „die Stimme unter dem Wasser“ bezeichnet wurde – hielt dort primitive Rituale ab, um das Wesen zu besänftigen. Mit dem Aufkommen des Klosters wurde der Kult verdrängt – jedoch nie ganz ausgelöscht.
Das Moor ist heute schwer zugänglich, doch bei Nebel oder zur Dämmerung wirken manche Wege wie verändert… fast so, als würden sie einen führen wollen.

\paragraph{1. Der Schreiende Stein}
\begin{itemize}
\item Lage: Ca. 1,5 Stunden Fußweg vom Stadtrand entfernt, nur über einen teilweise verfallenen Bohlenweg zu erreichen
\item Aussehen: Großer, glattgeschliffener Monolith (ca. 2,5 m hoch), umgeben von sumpfigem Tümpel
\item Merkmal: Vertikale Risslinie in der Mitte – pfeift bei starkem Wind, klingt wie ein Schrei
\item Funktion: Antike Opferstätte. Das Pfeifen galt als „Antwort Azaluths“
\item Spielleiter-Info: Bei Nacht oder Dämmerung kann der Stein schwach leuchten (biolumineszierendes Moos oder übernatürlich?)
\item Entdeckung: Eine alte Eisenfibel liegt halb versunken – sie zeigt das Symbol des alten Kults (Kreis mit durchbrochener Spirale)
\end{itemize}
    
\paragraph{2. Die Versunkene Halle}
\begin{itemize}
\item Lage: Unter Wasser verborgen, ca. 2 m unter der Oberfläche eines stillen Beckens mitten im Moor
\item Zugang: Nur bei starkem Trockenstand sichtbar – oder durch Hinweise auffindbar
\item Struktur: Mauerreste und eine halbversunkene Kuppel. Innen: Basreliefs, alte Kultinschriften (unverständlich, wirken hypnotisch)
\item Besonderheiten:
	\begin{itemize}
    \item Tiefe Furchen auf dem Boden → deuten auf wiederholte „Kreisbewegungen“
    \item Alte Eisenketten → jemand (oder etwas) wurde angebunden
    \item Auf einem Steinaltar: verblasste Zeichnung eines Auges mit sieben Pupillen
	\end{itemize}
\item Effekt:
	\begin{itemize}
    \item Wer länger als 10 Minuten bleibt, muss eine W100-Probe auf Geistige Stabilität ablegen (Schwierigkeit: 60)
    \item Misserfolg: Spieler hört „die Stimme“ → leise Bitten, Warnungen, Versprechen
	\end{itemize}
\end{itemize}

\paragraph{3. Das Bootshaus des Blinden Fischers}
\begin{itemize}
\item Lage: Abseits vom Hauptpfad, kaum sichtbar, halb in den Morast gesunken
\item Zustand: Baufällig, schiefer Turm, morsche Stege, Nebel hängt immer über dem Dach
\item Legende: Ein blinder Fischer lebte hier allein, sprach nie – bis man nur noch seine Stimme hörte
\item Fundstücke:  
	\begin{itemize}
    \item In einer Truhe: ein Tagebuch mit wasserverblasster Tinte – einzelne Sätze lesbar:
		\begin{itemize}
         \item „...er sah mich an, obwohl er keinen Kopf hatte...“
         \item „...sie flüsterten, aber meine Ohren waren voller Wasser...“
		\end{itemize}
    \item Unter dem Bootsboden: Runenplatte mit versiegeltem „Namen“ (verschmiert, nur „AZ...TH“ lesbar)
	\end{itemize}
\item Spielleiter-Info: Hier kann ein „Echo“ erscheinen – ein geisterhaftes Bild des Fischers, der auf dem Steg sitzt, rückwärts redet und sich bei genauer Betrachtung langsam im Kreis dreht
\end{itemize}

\paragraph{4. Der Totenpfuhl}
\begin{itemize}
\item Lage: Abgelegene, schwarze Wasserfläche, die nie zufriert
\item Besonderheit: Keine Tiere in der Nähe, keine Insekten, keine Geräusche – unnatürliche Stille
\item Funktion: Der Totenpfuhl war eine Ritualstätte, in der lebende Opfer versenkt wurden
\item Fundmöglichkeit:
	\begin{itemize}
    \item An einem verknoteten Ast hängt ein altes Amulett (ein Kreis mit eingravierter Spirale)
    \item Taucher oder wagemutige Spieler können bei guter Probe (Schwimmen + Mut) ein Skelett bergen, das ein metallisches Kästchen hält
    \item Inhalt: Wachsversiegelter Zettel mit Fragment eines Bannspruchs gegen Azaluth („...schweig, o namenlos...“)
	\end{itemize}
\end{itemize}

\paragraph{5. Der Steinkreis „Siebenmünde“}
\begin{itemize}
\item Lage: Auf einer kleinen trockenen Erhebung mitten im Moor (nur über schmalen Holzsteg erreichbar, der teilweise unter Wasser steht)
\item Struktur: 7 stehende Steine mit eingeritzten Symbolen, die sich zu drehen scheinen, wenn man sie nicht anschaut
\item Funktion: Alte Versammlung der Kultführer – jede Richtung des Kreises „blickt“ auf einen anderen Teil des Moores
\item Besonderheit:  
	\begin{itemize}
    \item In der Mitte eingelassen eine Schale – mit Resten eines schwarzen, versteinerten Harzes
    \item Wer das Harz berührt, sieht kurze Visionen (alte Rituale, Wasserwände, Form ohne Gestalt)
	\end{itemize}
\item Mechanik: Spieler*innen können hier durch ein Ritual (Okkultismus-Probe) Fragmente der Geschichte erlangen oder sogar das Wesen kurz spüren
\item Hinweis für SL:
	\begin{itemize}
    \item Diese Stelle ist sehr mächtig. Hier kann der Höhepunkt stattfinden (Öffnung eines Dimensionsrisses o. ä.)
	\end{itemize}
\end{itemize}

\subsection*{Hinweise für den Spielleiter}
\begin{itemize}
\item Der „Kult“ ist kein aktiver Club mit Roben, sondern ein Rest-Bewusstsein, das über Rituale in den Köpfen Einzelner lebt.
\item Spuren sind nie eindeutig: Der seltsame Stein könnte auch ein altes Grabmal sein. Das Bootshaus kann leer oder voller Flüche sein.
\item Verstärke psychologische Effekte statt Monsteraufmärschen: Träume, Wassergeräusche, Spiegelungen, Zweifel.
\item Karten-Logik: Gib den Spielern einen groben Plan („es gibt drei Orte“) – aber sie finden mehr, wenn sie frei suchen.
\end{itemize}