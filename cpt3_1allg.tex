\chapter{Das Szenario}
\section{Das Schweigen von Uhlenburg - Allgemeine Infos}
\subsection{Metadaten}
\paragraph{Setting}
Genre: Mysteriöses Horror-Drama mit subtiler Übernatürlichkeit\\
\textbf{Ort:} Die abgelegene Stadt Uhlenburg, nahe einem verfallenen Kloster und einem unheilvollen Moor\\
\textbf{Zeit:} Ca. 1783, Heiliges Westrheinisches Reich (fiktiv), nach einem harten Winter\\
\textbf{Ton:} Düster, geheimnisvoll, mit gelegentlichen absurden / schwarzhumorigen Momenten
\paragraph{Heldenrunde}
Die Spielergruppe ist zusammengesetzt aus Menschen (in dieser Welt existieren zwar Dunkle Mächte, Dämonen, Geister usw. aber diese werden als Aberglaube, Gerüchte und Gruselgeschichten abgetan).\\
Die Gruppe sollte aus \textbf{3 bis 4 Helden} bestehen die ihre eigene Persönlichkeit, Hintergrund (soz. Status, Beruf, Hobby, Fähigkeiten) haben. Diese müssen durch den SL sinnvoll zusammengeführt in diese Stadt gebracht werden. Wichtig ist, dass sie \textbf{Außenstehende} in der Stadt sind.
\paragraph{Grundidee}
Ein vergessener Kult verehrt ein „Älteres Wesen“, das in den Tiefen des Moores ruht. \\
Die Kreatur ist kein klassischer Dämon, sondern ein archaisches Bewusstseinsfeld aus vorzeitlicher Zeit (inspiriert von Folklore + kosmischem Schrecken).\\
Durch Rituale wird es schwächer gebunden – und die Fäden beginnen sich nun zu lösen.\\
Die Bevölkerung spürt „etwas“, aber erklärt es sich mit Sumpfgas, Wahnsinn oder Gottesprüfung. Es verschwinden Leute, Träume sind seltsam, Tiere benehmen sich merkwürdig.\\
Der letzte Abt des Klosters ist tot, und mit ihm ein Siegel, das das Wesen schwächte.\\
Der Kult war einst organisiert, ist heute aber nur noch als Erinnerung, Tradition, Wahnidee oder „Ziehen im Innern“ präsent. Frater Dorian ist keine klassische Führungsfigur, sondern ein Ankerpunkt, durch den Azaluth wieder greifbar wird. SL kann dies gezielt in Dialogen oder Visionen thematisieren.
\paragraph{Notwendiges}
Das Basis HTBAH macht die Proben mit einem W100 (idR ein W10 mit 1 bis 10 und ein W10 mit 00 bis 90).\\
Zusätzlich sind auch andere Würfel sinnvoll um dinge wie Schaden Wahnsinn usw. zu würfeln, sodass jedem Spieler ein vollständiger Satz Würfel W2, W4, W6, W8, W10, W20 bereitstehen sollte.\\
Das Modul Mentale Stabilität und Wahnsinn ist nötig bei diesem Szenario um insb die Reaktion auf Kontakt mit dem Übernatürlichen und Angst zu handeln.

\newpage
\subsection{Aufbau}
\subsubsection{Phase 1 - Ankunft / Erster Kontakt}
\begin{itemize}
\item Spieler*innen treffen sich in Uhlenburg (Auftrag, Erbe, Flucht, Neugier)
\item Treffen erste NSCs, hören Gerüchte (Spielleiter entscheidet, welche stimmen)
\item Optional: Erkundung von Stadt + Randgebieten
\end{itemize}
\subsubsection{Phase 2 - Spuren / Hinweise}
\begin{itemize}
\item Spieler*innen entdecken Karten, alte Briefe, Artefakte, alte Klosterpläne
\item Hinweise auf einen „vergrabenen Schrein“ und vergessene Namen in alten Chroniken
\end{itemize}
\subsubsection{Phase 3 - Enthüllung / Eskalation}
\begin{itemize}
\item Kulthandlungen nehmen zu (ein Kind verschwindet, Tiere ertrinken sich)
\item Showdown in alten Gemäuern / dem Schrein im Moor, Konfrontation mit Kult, Kreatur oder beidem
\end{itemize}
\subsection{SL-Hinweise}
\paragraph{Spielmechanik}
\begin{itemize}
    \item Verwende für wichtige Proben W100.
    \item Mach soziale Interaktion zu einem relevanten Teil:
    \begin{itemize}
        \item Beispiel: Spieler müssen NSCs überzeugen, helfen zu graben / Weg zu zeigen
    \end{itemize}
    \item Einbindung: Spieler können Hinweise sammeln wie in einem Detektivabenteuer, z.B.:
    \begin{itemize}
        \item Karte des Moores mit verblasstem Pfad
        \item Abgebrannte Seite eines Tagebuchs mit Rezept für Bannsalz
        \item Kräutermischung, die bei Ritualen stört
	\end{itemize}
    \item Wähle, wie real die Bedrohung wird
    \begin{itemize}
        \item Willst du ein klassisches „Kult stoppt Erwachen“ oder ein Lovecraftian-Szenario mit offenem Ende?
	\end{itemize}
    \item Nutze Musik und Geräusche: Tropfende Wände, krächzende Raben, Flüstern im Wind
    \item Baue falsche Fährten ein:
    \begin{itemize}
        \item Alte Schuld unter Bürgern, die nichts mit dem Kult zu tun hat
        \item Streit um Erbe, der ablenkt
	\end{itemize}
    \item Optionaler Humor:
    \begin{itemize}
        \item Die Wirtin gibt jedes Getränk in der Suppe aus (Suppe mit Rum, Suppe mit Bier...)
        \item Der Bürgermeister kann keine Zahlen lesen, lässt sich aber nichts anmerken
	\end{itemize}	
\end{itemize}
\newpage
\subsection{Charakterbau}
Beim Charakterbau sind alle Möglichkeiten offen. Worauf verzichtet werden sollte, ist Magier und ähnliche klassische high Fantasy Klassen zu wählen. Wir wollen hier \textbf{normale} Charakere in eine \textbf{abnormale} Situation werfen. Dennoch dürfen diese natürlich auch gewisse Eigenarten haben, die dem Spielverlauf förderlich sind. Dazu gehören insb. auch wissen Über Okultes. Sollte das gar nicht zu den Spielercharateren passen muss der SL versuchen durch finden eines Buches / Briefe / Artefakte dieses harmonisch an die Spieler zu verteilen oder den Spielern anders diese Fähigkeiten beibringen.
\subsubsection*{Fähigkeiten-Matrix}
Dies hilft bei der Erstellung oder Anpassung von Charakteren sowie bei der Entscheidung, welche Fähigkeiten belohnt, welche ignoriert und welche sogar zu gefährlichen Situationen führen können.
\paragraph{Essentiell (Plotrelevant)}
Wahrnehmung, Willenskraft, Okkultismus, Intuition\\
Diese Fähigkeiten ermöglichen das Erkennen von Hinweisen, Überleben mentaler Belastung und Verstehen der Kultlogik / Rituale
\paragraph{Nützlich (optional stark)}
Heimlichkeit, Menschenkenntnis, Zeichnen, Wissen (Mythen/Kirche), Kräuterkunde\\
Helfen bei alternativen Wegen (z. B. schleichen statt kämpfen), dem Deuten von NSC-Lügen oder dem Verstehen symbolischer / ritueller Elemente
\paragraph{Unnütz (stimmungsvoll, aber gameplay-neutral)}
Reiten, Alchemie, Fälschen, Handwerk (außer thematisch relevant)\\
Können für Rollenspiel genutzt werden, tragen aber nicht aktiv zum Lösen des Plots bei (es sei denn, SL improvisiert sinnvollen Einsatz)
\paragraph{Schädlich / gefährlich}
Einschüchtern (NSCs ziehen sich zurück), Aggressiver Kampfdrang, rationales Abtun („Ich glaub nicht an sowas.“)\\
Können Situationen eskalieren lassen oder Hinweise verlieren lassen – führt ggf. zu früher Beschwörung, Toten, Wahnsinn, Fehleinschätzungen

\subsubsection*{Anwendung im Spiel}
Beispielhafte SL-Entscheidung:\\
Ein SC hat hohe „Wissenschaft (Botanik)“ und untersucht Pflanzen am Steinkreis.\\
→ Nutze das als „Hilfreich“: Lass sie erkennen, dass die Pflanzen „kreisförmig gewachsen sind, als würde etwas unsichtbares Einfluss nehmen“.
Nicht essentiell, aber immersiv und belohnend.

Beispielhafte Gefahr:\\
Ein SC benutzt Einschüchtern gegen Gerlinde, um sie zur Herausgabe eines alten Buches zu zwingen.\\
→ Schädlich: Sie schließt sich ein, bricht mental zusammen – der Hinweis (Bannseite) wird nicht auffindbar. Vielleicht sogar Reaktion des Kults.

\newpage
\subsection{SL-Leitblatt}
\paragraph{Wichtige Hinweise}
\begin{center}
\begin{tabular}{llll}
\toprule
Wichtig & Hilfreich & Irreführend & Unwichtig\\
\midrule
Adalberns Brief & Lises Zeichnungen & Rudels Predigten & Uhrzeit der Messe\\
\midrule
Bannformel (wenn vollständig) & Karte mit Moorstellen & Tierverhalten im Dorf & Wer den Wein\\
&&& geklaut hat\\
\midrule
Siegelkachel & Kräuterlied der Magda & Aussage Ulrichs über & Alte Steuerlisten\\
&&„sprechende Bäume“ &\\
\midrule
Kultamulett & Bootshaustagebuch & Vater Rudel beschuldigt Lise & Glockenläuten\\
\bottomrule
\end{tabular}
\end{center}

\paragraph{Zufallsereignisse \& Visionstabellen}
\subparagraph{1W6 - Visionen bei Berührung alter Relikte}
\begin{enumerate}
\item Du siehst das Moor aus der Sicht eines toten Tieres – flackernd, verzerrt
\item Du hörst deinen eigenen Namen rückwärts geflüstert
\item Eine Tür, hinter der du selbst stehst
\item Wasser tropft von innen an deine Haut
\item Ein steinerner Kreis – du weißt, wo er ist, obwohl du nie dort warst
\item Ein Lied, das du nie gelernt hast, das aber in deinem Kopf „auf dich wartet“
\end{enumerate}
\subparagraph{1W6 – Albträume in der Nacht}
\begin{enumerate}
\item Du kannst dich nicht erinnern, wer du bist – nur der Name „Azaluth“ bleibt
\item Jemand ruft aus dem Wasser – mit deiner Stimme
\item Du öffnest ein Tür – hinter dir
\item Ein schwarzer Vogel schreit sieben Mal, dann wird alles still
\item Du wachst auf. Im Traum. Wieder. Und wieder
\item Du stirbst – aber jemand anders steht auf
\end{enumerate}

\subparagraph{Wahnsinns-Effekte (Wurf auf Willenskraft misslungen)}
\begin{center}
\begin{tabular}{ll}
\toprule
Effekt & Dauer\\
\midrule
Flüstern hören, das niemand sonst hört & 10 Minuten\\
\midrule
Kurzzeitige Amnesie (Name, Ort) & 1 Stunde\\
\midrule
Vision: Körper hat keine Haut mehr & 1 Szene\\
\midrule
Du glaubst, du seist schon gestorben & 2 Szenen\\
\midrule
Du kannst Gesichter nicht mehr erkennen & Bis zum Schlaf\\
\midrule
Du glaubst, du sprichst, aber kein Ton kommt & 15 Minuten (Stresswurf möglich)\\
\bottomrule
\end{tabular}
\end{center}
\newpage
\paragraph{Finale Szene – Auslösende Faktoren}
\subparagraph{Der Kult startet das Ritual bei einem dieser Trigger}
\begin{itemize}
\item Spieler greifen Kultisten im Steinkreis an
\item Lise wird verletzt oder getötet
\item Amulett zerbricht
\item Name „Azaluth“ wird laut ausgesprochen
\end{itemize}
\subparagraph{Spielende-Optionen}
\begin{itemize}
\item Ritual gelingt → die Welt verändert sich subtil (kein Komet, sondern „Wahrnehmung“)
\item Ritual scheitert → Azaluth „vergisst“ Uhlenburg für nun
\item Spieler opfern Erinnerung, Leben oder ihren Namen → Welt bleibt stabil
\item Kompromiss: Kult lebt weiter, aber verborgen – SCs wissen, dass es nie ganz endet
\end{itemize}