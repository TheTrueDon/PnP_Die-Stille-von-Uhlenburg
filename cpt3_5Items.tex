\section{Gegenstände}
\subsection*{1. Der Brief des Mönchs Adalbern}
\paragraph{Fundort:} Im Turmarchiv der alten Stadtmauer, zwischen zwei Pergamentrollen
\paragraph{Material:} Blasse Tinte auf rauem Papier, leicht verbrannt am Rand\\
\begin{displayquote}
„An die ehrwürdige Bruderschaft,
ich fürchte, unsere Ketten haben versagt. Der Stein singt wieder, das Wasser spricht, und Bruder Emeran spricht in Sprachen, die der Heilige Augustinus nicht kennt.
Ich habe das Siegel versenkt, möge es dort ruhen. Wer es hebt, trage die Schuld.
Azaluth regt sich. Der Kreis bricht.“\\
\textit{Adalbern, im 7. Monat des Jahres 1172}
\end{displayquote}
\paragraph{Spielleitung:}
\begin{itemize}
\item Deutet auf das versiegelte Artefakt im Moor
\item Datum zeigt, dass das Problem über 600 Jahre alt ist
\item Könnte Rückschlüsse auf „Siegelorte“ oder Kultzeichen zulassen
\end{itemize}

\subsection*{2. Ritualtisch-Beschreibung (Kulttext)}
\paragraph{Fundort:} In der Krypta unter der Klosterruine, eingeritzt in eine Steinplatte
\paragraph{Format:} Formloser Text ohne Absätze – wirkt wie ein Mantra
\begin{displayquote}
„Sieben auf Stein\\
Wasser im Kreis\\
Schweigend nennen\\
Nicht das Herz\\
Nicht den Ort\\
Nur den Klang\\
Lass ihn sein.“
\end{displayquote}
\paragraph{Effekt:}
\begin{itemize}
\item Spieler könnten denken, es handle sich um einen Zauberspruch
\item In Wahrheit: Anleitung für ein Schweigeritual zur Verstärkung der Entität
\end{itemize}

\subsection*{3. Unvollständige Karte des Moors}
\paragraph{Fundort:} Bürgermeisterhaus, in einer Truhe unter Akten über Brückenreparaturen
\paragraph{Format:} Pergament, grobe Umrisse, nur drei markante Stellen eingezeichnet:
\begin{itemize}
\item Der „Schreiende Stein“ (mit rotem X)
\item „Siebenmünde“ (aber ohne Namen)
\item Ein durchgestrichenes „Haus des Fischers“
\end{itemize}
\paragraph{Randnotiz:}
\begin{displayquote}
„Achtung: Brücke Nordwest unpassierbar.\\
Kein Zugang bei hohem Wasserstand.\\
--- V.“
\end{displayquote}
\paragraph{Verwendung:}
\begin{itemize}
\item Hilft den Spielern beim Auffinden von Kultstätten
\item Lässt offen, wie viele Orte es noch gibt
\item Hinweis auf gefährlichen Zugang („Nordwest unpassierbar“) ist relevant fürs Finale
\end{itemize}

\subsection*{4. Tagebuchseiten aus dem alten Bootshaus}
\paragraph{Fundort:} In einem versiegelten Fach unter einer losen Diele
\paragraph{Material:} Stark angegriffenes Notizbuch, Seiten teilweise verklebt
\begin{displayquote}
„Ich hör sie wieder.\\
Der Nebel kommt nicht mehr nur von außen.\\
Letzte Nacht war er im Zimmer.“\\
„Ich hab das Seil verbrannt. Es hat gezischt wie ein Fisch.“\\
„Er spricht nicht. Aber wenn ich mein Ohr ins Wasser lege… dann weiß ich, wie ich sterbe.“
\end{displayquote}
\paragraph{SL-Notiz:}
\begin{itemize}
\item Spätere Seiten bestehen nur aus wiederholten Zeichen
\item Option: Spieler mit Okkultismus > 60 können erkennen, dass dies „Schrift der formlosen Dinge“ genannt wurde (rein ritualistisch)
\end{itemize}

\subsection*{5. Kult-Amulett (Artefakt)}
\paragraph{Fundort:} Um den Hals eines Skeletts im Totenpfuhl
\paragraph{Aussehen:} Kreisförmig, aus schwarzem Stein mit eingelegter Spirale in Bronze
\paragraph{Effekt (optional):}
\begin{itemize}
\item Träger erhält bei Berührung leichte Halluzinationen (optisch)
\item Gibt Vorteil bei Erkennung von Kulttexten (+20 auf Okkultismus-Proben)
\item Kann als Schlüssel für die Runentafel im Steinkreis „Siebenmünde“ dienen
\end{itemize}
\paragraph{Gefahr:}
Wird das Amulett während eines Rituals zerbrochen, zieht es Azaluths „Blick“ auf den Träger → SL entscheidet: Wahnsinn? Albträume? Körperveränderung?

\subsection*{6. Bannformel (Fragment auf versiegeltem Pergament)}
\paragraph{Fundort:} In einem Wachskästchen in Elias' Hütte, gut versteckt unter Knochen
\paragraph{Material:} Dünnes Pergament, fast durchscheinend. In lateinähnlicher Sprache:
\begin{displayquote}
„...Et silere nomen eius\\
non laudare, non rufen,\\
sed vincire aquam et tonum...“
\end{displayquote}
\begin{displayquote}
(Deutsche Randübersetzung, vermutlich von Elias:)\\
„...und schweige seinen Namen.\\
Nicht loben, nicht rufen,\\
sondern binden Wasser und Ton.“
\end{displayquote}
\paragraph{Verwendung:}
\begin{itemize}
\item Wenn vollständig entschlüsselt (vielleicht durch mehrere Funde), kann dies das große Ritual abschwächen oder „neutralisieren“
\item Kann Spieler glauben lassen, es gäbe eine Möglichkeit zur Rettung
\end{itemize}

\subsection*{7. Lises Zeichnungen}
\paragraph{Fundort:} Im Kinderzimmer von Lise, an die Wand geklebt
\paragraph{Motiv:}
\begin{itemize}
\item Ein Haus mit vielen Türen, aber keine Fenster
\item Eine Figur mit 7 Armen, die im Wasser steht
\item Ein Gesicht, das sich beim zweiten Blick verändert (z. B. bekommt mehr Augen)
\end{itemize}
\paragraph{Spielerreaktion:}
\begin{itemize}
\item Spieler könnten sie abschreiben, um sie zu analysieren
\item Optionaler Effekt: Wer ein Bild bei sich trägt, träumt dieselbe Nacht davon
\item Psychologisches Stilmittel: Zeigt, wie Azaluth in Kinderträume gelangt
\end{itemize}

\subsection*{8. Die Siegelkachel}
\paragraph{Fundort:} Unter dem Altar der Klosterruine, versiegelt mit rotem Harz
\paragraph{Beschreibung:}
\begin{itemize}
\item Kleine Steintafel, die bei Berührung warm wird
\item Symbol: Kreis in Spirale mit sechs Strichen → der „Zirkel der Stillen Mündung“
\end{itemize}
\paragraph{Spielmechanik:}
\begin{itemize}
\item Kann genutzt werden, um an einem der Ritualorte Azaluth kurzfristig zu binden (z. B. verhindert, dass jemand geopfert wird)
\item Zerbricht bei übermäßigem Gebrauch → dann ist der „Schutz“ verloren
\end{itemize}

\newpage
\subsection{Übersicht: Wichtige Gegenstände \& Hinweise – Fundorte und Zugänge}
\paragraph{Adalberns Brief}
Turmarchiv der alten Stadtmauer\\
Finden zwischen Pergamentrollen; SCs müssen Archivar oder Archiv gut durchsuchen, evtl. Rätsel lösen, Klettern oder Schlüssel finden
\paragraph{Fragment der Bannformel}
Dachbalkenritze bei Gerlinde Vogt\\
Gerlinde besitzt das Fragment heimlich; Überzeugen/Vertrauen aufbauen, ggf. kleines Gefallen erfüllen, um es zu erhalten
\paragraph{Unvollständige Moor-Karte}
Bürgermeisterhaus in Truhe\\
Zugang über Bürgermeister oder Verwaltungsbeamte, evtl. Überreden oder heimliches Finden durch Einbruch
\paragraph{Tagebuchseiten aus Bootshaus}
Versiegeltes Fach unter Dielen im Bootshaus\\
Spieler finden bei genauer Suche, evtl. Hebel, versteckte Klappe oder Nachfragen beim Bootshausbesitzer (z. B. Ulrich)
\paragraph{Kult-Amulett}
Um den Hals eines Skeletts im Totenpfuhl im Moor\\
Spieler müssen das Moor erkunden, ggf. gefährliches Terrain durchqueren; Fund durch Suche an den richtigen Stellen oder Hinweise von NSCs (z. B. Marius)
\paragraph{Siegelkachel}
Unter dem Altar in der Klosterruine\\
Zugang nur durch Erforschen der Ruine, evtl. Entschärfen von Fallen oder lösen von Rätseln, ggf. Hinweise von Magda oder Frater Dorian
\paragraph{Lises Zeichnungen}
Kinderzimmer von Lise\\
Lise zeigt die Zeichnungen freiwillig oder nach Vertrauenserwerb; Spieler müssen Lise schützen und beruhigen, um Zugang zu bekommen
\paragraph{Bannformel (vollständig)}
Kombination von Gerlindes Fragment + Pergament in Elias’ Hütte\\
Zusammensetzen erfordert Sammeln beider Fragmente; Einholen beim geheimnisvollen Elias (kann auf Wunsch gespielt oder als NSC auftauchen)
\paragraph{Bannlied (halbe Strophe)}
Von Magda „die Mager“\\
Spieler erhalten durch Gespräche mit Magda, wenn sie Vertrauen aufbauen und Interesse an Ritualen zeigen
\paragraph{Bootshaustagebuch (zusätzliche Hinweise)}
Bootshaus, evtl. von Ulrich oder versteckt\\
Ulrich könnte Widerstand leisten, wenn man nicht vorsichtig vorgeht; Spieler können mit ihm verhandeln oder das Buch heimlich finden
\newpage
\subsection{NSC-Übergaben \& Interaktionsübersicht}
\paragraph{Gerlinde Vogt}
Fragment der Bannformel (Pergament)\\
Bedingung: 
Vertrauen aufbauen, kleine Gefallen erfüllen, z. B. Besorgungen erledigen\\
Tipp: 
Höflichkeit, Zuhören, Mitgefühl zeigen
\paragraph{Magda „die Mager“}
Bannlied (halbe Strophe)\\
Bedingung: 
Interesse an Ritualen zeigen, Geduld beim Gespräch\\
Tipp: 
Fragen nach Kräutern und alten Bräuchen
\paragraph{Lise}
Zeichnungen mit Visionen\\
Bedingung: 
Schutz bieten, Freundschaft schließen, sie beruhigen\\
Tipp: 
Rücksicht nehmen, Fragen zu ihren Bildern stellen
\paragraph{Bürgermeister / Archivar}
Moor-Karte, Adalberns Brief\\
Bedingung: 
Überreden oder durchsuchen, evtl. mit Bestechung oder diplomatischem Takt\\
Tipp: 
Diplomatie, Wissensproben (Lokal)
\paragraph{Ulrich (Bootshaus)}
Tagebuchseiten Bootshaus\\
Bedingung: 
Überzeugen oder heimliches Finden, evtl. Handel anbieten\\
Tipp: 
Vorsicht vor Konfrontation, alternative Wege finden
\paragraph{Elias (Geheimnisvoller Helfer)}
Fragment Bannformel / Hinweise\\
Bedingung: 
Findet man ihn in der Ruine oder Moor, muss Vertrauen aufgebaut werden\\
Tipp: 
Respektvoll, geduldig, bereit zu teilen
\paragraph{Frater Dorian}
Hinweise zum Kult und Ritual\\
Bedingung: 
Nur bei diplomatischem oder opferbereitem Verhalten\\
Tipp: 
Ruhe bewahren, keine offene Konfrontation

